\section{Zusammenfassung und Schlussfolgerungen}

In dem vorliegenden Projektbericht wird eine Methode zur Klassifizierung von Retina Aufnahmen, welche durch die optische Kohärenztomographie (OCT) erstellt werden, basierend auf Methoden des überwachten maschinellem Lernen. Die Aufnahmen sind dabei in vier Klassen unterteilt. Drei dieser Klassen weisen Erkrankungen der Retina auf, während die vierte Klasse Aufnahmen zeigen bei denen keine der drei Erkrankungen vorliegen. \\
Die vorgestellte Methode baut auf den sogenannten tiefen faltenden Netze (CNN) auf. Eine Struktur, die sich aus faltenden Lagen, Aggregationsschichten und dichten Lagen bei Verwendung des sogenannten Dropouts sehr gute Ergebnisse liefert. Es wird gezeigt, dass der Lernerfolg des CNN erheblich gesteigert wird, wenn die Zusammensetzung des Datensatzes in Form von Gewichten so berücksichtigt wird, dass jede der Klassen während des Trainings gleich behandelt wird. Zudem kann der Lernerfolg des Netzes durch die Reduzierung der Dimension der dichten Lagen gesteigert. Aufgrund der langen Laufzeit eines Trainingsprozesses kann die Konfiguration des CNN nicht durch eine größer angelegte Optimierungsstudie optimiert werden, jedoch liefert die in diesem Bericht vorgestellte finale Struktur bereits sehr gute Ergebnisse und erreicht eine Gesamtgenauigkeit von $90\,\%$.  \\
Die Wahl eines CNN wird dadurch validiert, dass der Lernerfolg eines vollständig vernetzten neuronalen Netzes (DNN) auf dem selben Datensatz ermittelt wird. Zu diesem Zweck wird die Struktur des Netzes, sowie die Aktivierungsfunktionen und die Batch Größe optimiert. Es stellt sich dabei heraus, dass kleinere Batch Größen bessere Ergebnisse erzählen. Die höchste Genauigkeit, die jedoch erzielt werden kann, liegt bei $72\,\%$ und ist somit deutlich geringer als die des CNN. Daher wird deutlich, dass die Wahl eines CNN für die gewählte Aufgabenstellung OCT Aufnahmen der Retina zu klassifizieren am besten geeignet scheint. \\