\section{Zusammenfassung und Schlussfolgerungen}

In dem vorliegenden Projektbericht wird eine Methode des Maschinellen Lernens zur Klassifizierung von Erkrankungen der Retina vorgestellt, indem Aufnahmen der optischen Kohärenztomographie (OCT) analysiert werden. Die Aufnahmen beinhalten dabei drei unterschiedliche Erkrankungen der Retina und Aufnahmen, bei denen keine dieser Erkrankungen vorliegt.\\
Die vorgestellte Methode verwendet ein tiefes faltendes neuronales Netz (CNN), wobei eine Struktur gewählt wird, die sich aus faltenden Lagen, Aggregationsschichten und vollständig vernetzten dichten Lagen bei Verwendung des sogenannten Dropouts sehr gute Ergebnisse mit einer Genauigkeit von $90\,\%$ liefert. Es wird gezeigt, dass der Lernerfolg des CNN erheblich gesteigert wird, wenn die Zusammensetzung des Datensatzes in Form von Gewichten berücksichtigt wird, sodass jede der Klassen während des Trainings gleich behandelt wird. \\
Das Training des CNN gestaltet sich jedoch als sehr zeitaufwendig, sodass eine vollständige Optimierung des Netzes nicht vollzogen wird. Es wird jedoch gezeigt, dass die Reduzierung der Dimension der dichten Lagen eine etwas verbesserte Genauigkeit liefert. Zudem wird dadurch die Trainingsdauer reduziert, sodass in weiterführenden Studien eine weitere Reduktion der Dimension dieser Lagen vielversprechend erscheint. \\
Darüber hinaus wird festgestellt, dass die elu (exponential linear unit) als Aktivierungsfunktion eine etwas bessere Genauigkeit liefert als die relu (rectified linear unit) Funktion. Beim Training über 40 Epochen wird kein Übertraining festgestellt, sodass eine Optimierung in Hinblick auf die Verminderung des Übertrainings nicht erfolgen muss. Außerdem kann die Dimension vor der ersten flachen vollständig vernetzten dichten Lage noch weiter reduziert und der Lernerfolg eventuell gesteigert werden, indem weitere Conv2D oder Pooling Lagen hinzugefügt werden. \\
Die Wahl eines CNN wird dadurch validiert, dass der Lernerfolg eines optimierten flachen neuronalen Netzes (DNN) auf dem selben Datensatz ermittelt wird. Zu diesem Zweck wird die Struktur des Netzes, sowie die Aktivierungsfunktionen für die versteckten Lagen und die Ausgangslage sowie die Batch Größe optimiert. Es stellt sich dabei heraus, dass kleinere Batch Größen bessere Ergebnisse erzielen. Die höchste Genauigkeit, die erzielt werden kann, liegt hier bei $72\,\%$ und ist somit deutlich geringer als die des CNN. Daher wird deutlich, dass die Wahl eines CNN für die gewählte Aufgabenstellung OCT Aufnahmen der Retina zu klassifizieren besser geeignet ist als ein DNN. \\
Abschließend lässt sich demnach schlussfolgern, dass die hier gewählte Struktur des CNN im Rahmen dieser Projektarbeit mit einer Genauigkeit von $90\,\%$ ein zufriedenstellendes Ergebnis liefert. \\
