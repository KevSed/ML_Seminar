\section{Zusammenfassung und Schlussfolgerungen}

In dem vorliegenden Projektbericht wird eine Methode zur Klassifizierung von Retina Aufnahmen, welche durch die optische Kohärenztomographie erstellt werden, vorgestellt, die auf Maschinellem Lernen basiert. Hierbei wird zwischen 4 Klassen von Aufnahmen unterschieden. Es gibt drei Klassen, welche Augenkrankheiten darstellen, die ... . Zur vierten Klasse werden alle Aufnahmen gezählt, die keine der Symptome der drei Krankheiten zeigt. \\
Es wird gezeigt, dass durch ein neuronales Netz, welches aus vollständig vernetzten Faltungs- und dichten Lagen, sowie sogenannte Pooling Lagen, die eine Dimensionsreduktion des Bildes vornehmen ein sehr gutes Ergebnis liefert. Es wird eine Genauigkeit von $90\,\%$ erzielt, wobei die Klassen am stärksten und die Klassen am schwächsten getrennt werden. Es stellt sich heraus, dass durch die Berücksichtigung der unterschiedlichen Klassengröße durch Gewichte in der Verlustfunktion bessere Ergebnisse in der Klasse und eine bessere Genauigkeit erzielt werden kann. \\
Um die Methodik zu validieren wird eine alternative Herangehensweise vorgestellt, die auf einem neuronalen Netzen aus vollständig vernetzten dichten Lagen basiert. Hierbei stellt sich heraus, dass eine grobe Verpixelung und somit eine optimierte Laufzeit zu keinem signifikanten Verlust in der Genauigkeit darstellt. Daher ist diese Methodik deutlich schneller, als die vorgestellte . Es werden 120 Netzstrukturen getestet, wobei die beste Struktur eine Genauigkeit von $72\,\%$ liefert. Obwohl dieses Netz deutlich stärker auf mögliche Anpassungen der Netzstruktur untersucht wird, ist dieses Netz $18\,\%$ schlechter als die vorgestellte Methode. Demnach lässt sich schlussfolgern, dass die Wahl eines solchen Netzes am geeignetesten erscheint. \\
Obwohl das Netz schon eine sehr gute Klassifizierungsmöglichkeit der untersuchten Aufnahmen ermöglicht, kann die Struktur und die Parameter des Netzes noch verbessert werden. Dies ist durch die großen Menge an Bildern jedoch sehr zeitaufwendig. Jedoch wird bereits durch die getesten Strukturen und Parameter deutlich, dass keine signifikante Verbesserungen mehr zu erwarten sind, sodass die in diesem Bericht vorgestellte Struktur sehr gute Ergebnisse liefert und die Aufgabe der Klassifizierung der Retina Aufnahmen mehr als zufrieden stellend löst. 
