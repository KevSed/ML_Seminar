\section{Analysierter Datensatz}
% \begin{table}[!h]
% \centering
% \caption{Zusammensetzung des untersuchten Datensatzes aufgeteilt nach den im Datensatz vorhandenen Krankheiten, CNV, DME und DRUSEN, und Aufnahmen, die keine dieser Krankheiten aufweisen und mit NORMAL gekennzeichnet werden.}
% \label{tab:datacomp}
%  \begin{tabular}{|c|c|}
%  \hline
%  Erkrankung & Anzahl \\
%  \hline
%  NORMAL & 26315	\\
%  CNV &  37205 \\
%  DME &	11348	\\	
%  DRUSEN & 8616	\\
%  \hline
%  Total	& 83484\\
%  \hline
%  \end{tabular}
% \end{table}

 \begin{table}[!h]
\centering
\caption{Zusammensetzung des untersuchten Datensatzes aufgeteilt nach den im Datensatz vorhandenen Krankheiten, CNV, DME und DRUSEN, und Aufnahmen, die keine dieser Krankheiten aufweisen und mit NORMAL gekennzeichnet werden.}
\label{tab:datacomp}
\begin{tabular}{|c|cccc|c|} 
\hline
Erkrankung & NORMAL & CNV & DME & DRUSEN & Total \\
\hline
Anzahl &26315 & 37205 & 11348 & 8616 & 83484 \\
\hline
 \end{tabular}
\end{table}


Der verwendete Datensatz besitzt die in Tabelle \ref{tab:datacomp} dargestellte Zusammensetzung und in Abbildung \ref{fig:example} sind Beispielaufnahmen für jede der Klassen dargestellt. Die Aufnahmen sind in Graustufen aufgezeichnet, wodurch die Pixelwerte den Wertebereich $[0,\,255]$ abdecken, wobei schwarze Pixel dem Wert 0 und wei{\ss}e Pixel dem Wert 255 entsprechen. Die Erkrankung CNV ist in Abbildung \ref{fig:CNVexa} abgebildet und durch eine deutliche Schwellung der Retina zu erkennen. Abbildung \ref{fig:DMEexa} zeigt eine Retina, die von einem DME befallen ist, was an der ründlichen, schwarzen Färbung, die einer Gefä{\ss}ausbuchtung entspricht, innerhalb einer deutlichen Schwellung der Retina erkennbar ist. Auch DRUSEN sind in Abbildung \ref{fig:DRUSENexa} gut sichtbar, da unterhalb der Netzhaut deutliche Ausbuchtungen zu sehen sind, die im Vergleich zu der NORMAL Aufnahme in Abbildung \ref{fig:NORMexa} nicht erwartet werden. Es sollte jedoch hier herausgestellt werden, dass die Aufnahmen in Abbildung \ref{fig:example} ideale Beispiele darstellen, bei denen die Klassifizierung auch durch unerfahrene Betrachter vorgenommen werden kann. Bei vielen Aufnahmen ist dies jedoch nicht der Fall, sodass zur Klassifizierung eine andere Methodik oder Erfahrung in der Auswertung dieser Aufnahmen notwendig ist.  
\begin{figure}[!b]
\centering
 \begin{subfigure}[NORMAL\label{fig:NORMexa}]{
   \includegraphics[width=.20\linewidth]{fig/NORMAL.jpeg}}
 \end{subfigure}
 \begin{subfigure}[CNV \label{fig:CNVexa}]{
 \includegraphics[width=.20\linewidth]{fig/CNV.jpeg}}
 \end{subfigure}
 \begin{subfigure}[DME \label{fig:DMEexa}]{
   \includegraphics[width=.20\linewidth]{fig/DME.jpeg}}
 \end{subfigure}
 \begin{subfigure}[DRUSEN \label{fig:DRUSENexa}]{ 
   \includegraphics[width=.20\linewidth]{fig/DRUSEN.jpeg}}
 \end{subfigure}
 \caption{OCT Aufnahmen der Retina, die beispielhaft eine Aufnahme jeder untersuchten Klasse im Datensatz zeigen. Dabei zeigt Abbildung \ref{fig:NORMexa} eine OCT Aufnahme der Klasse NORMAL, die von keiner der Krankheiten, CNV, DME und DRUSEN, aufweist, welche in den Abbildungen \ref{fig:CNVexa}, \ref{fig:DMEexa} respektive \ref{fig:DRUSENexa} dargestellt sind.}
 \label{fig:example}
\end{figure}
\setcounter{subfigure}{0}

% \begin{table}
% \begin{tabular}{|c|c|c|c|c|}
%  \cline{1-1} \cline{3-3} \cline{5-5}
%   Conv2D (64, $(2\times 2)$, (2,\,2))& \multirow{2}{*}{$\Rightarrow$} & Pooling ($(3\times 3)$) & \multirow{2}{*}{$\Rightarrow$} & Conv2D (32, $(2\times 2)$, (2,\,2))\\
% \cline{1-1} \cline{3-3} \cline{5-5}
%  $(199\times 199 \times 64)$ & & $(66\times 66 \times 64)$ & & $(32\times 32 \times 32)$\\
% \cline{1-1} \cline{3-3} \cline{5-5}
%  & & & & $\Downarrow$ \\
%  & & & \multirow{2}{*}{$\Leftarrow$} & Pooling ($(3\times 3)$) \\
% \end{tabular}
% \end{table}




