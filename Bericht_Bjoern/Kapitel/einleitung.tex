\section{Einleitung}

Die Optische Kohärenztomographie (OCT) basiert auf dem Prinzip des Michelson-In\-ter\-fe\-ro\-me\-ters~\cite{OCT}. Kohärentes Licht, welches sich aus Wellenlängen einer festgelegten Bandbreite zusammensetzt, wird auf einen semitransparenten Spiegel gesendet. Dabei wird ein Teil des Lichtes transmittiert und der andere Teil reflektiert. Einer dieser entstandenen Teilstrahlen wird auf einen Spiegel gesendet, an dem er reflektiert und als Referenzstrahl benutzt wird. Der andere Teilstrahl wird auf die zu untersuchende Probe gesendet. Dringt dieser Teilstrahl in die Probe ein, so tritt er mit den Atomen und Molekülen des Probenmaterials in Wechselwirkungen. Er kann dabei gestreut, absorbiert oder reflektiert werden. Die Reflektion kann dabei in unterschiedlichen Schichttiefen geschehen. \\
Der reflektierte Referenzstrahl und der von der Probe reflektierte Strahl treffen an dem semitransparenten Spiegel wieder zusammen, wo beide in die gleiche Richtung transmittiert beziehungsweise reflektiert werden. Dabei interferieren die beiden Teilstrahlen und werden auf einen Detektor geleitet. Dieses analysiert die Intensität des Signals als Funktion der Wellenlänge. Aus der Form des Signals kann dabei ermittelt werden bei welcher Probentiefe die Reflexion des auf die Probe treffende Strahl vorliegt. Während der Analyse überlagern sich jedoch mehrere Signale, die durch Reflektion in verschiedenen Probentiefen entstehen, sodass diese zunächst getrennt werden müssen. Dazu kann die Fourier-Transformation genutzt werden, welche diese Signale trennt und somit direkt die komplette Information über das Tiefenprofil der Probe liefert. Durch Verschieben der Apparatur entlang der Probe können somit 2 oder 3 dimensionale Bilder der Probe erstellt werden. \\
Eines der Hauptanwendungsgebiete der OCT findet sich in der Medizin wieder. Es kann dazu genutzt werden die menschliche Retina auf Krankheiten zu untersuchen. Die Retina besitzt eine komplizierte, mehrschichtige Struktur und Erkrankungen der Retina können in jeder Schicht auftreten. Daher ist die Aufnahme eines Tiefenprofils der Retina von großem medizinischen Interesse. \\
In der vorliegenden Projektarbeit werden OCT Aufnahmen der Retina analysiert, wobei der Datensatz von Ref.~\cite{Dataset} verwendet wird. Die darin enthaltenen Aufnahmen lassen sich in 4 Klassen unterteilen. Während drei Klassen Aufnahmen beinhalten, welche Erkrankungen der Retina beinhalten, zeigt die vierte Klasse Retina Aufnahmen, die keiner der drei Erkrankungen aufweist und im Folgenden als NORMAL bezeichnet wird. Eine der Krankheiten ist die choroidale Neovaskularisation (CNV), wobei sich irreguläre Blutgefäße in der Netzhaut bilden, aus welchen Flüssigkeiten treten können, welche zu Schwellungen der Netzhaut führen. Drusen, DRUSEN im Folgenden, bezeichnen Abfallprodukten aus der zentralen Netzhaut, die sich unterhalb der Netzhaut ansammeln. Das diabetische Makulaödem (DME) ist eine Schwellung der Makula, die im Zuge der Bildung von kleinen Gefäßausbuchtungen in der Netzhaut entsteht.  \\
In dem Projektbericht wird eine Methode basierend auf überwachtem, maschinellem Lernen vorgestellt, die es ermöglicht diese OCT Aufnahmen zu klassifizieren. Diese Methode verwendet dabei ein tiefes faltendes neuronales Netz (CNN) und wird mit einer Methode verglichen, die auf ein neuronales Netz welches aus flachen dichten Lagen besteht (NN). 