\section{Einleitung}

Die Optische Kohärenztomographie (OCT) basiert auf dem Prinzip des Michelson-In\-ter\-fe\-ro\-me\-ters~\cite{OCT}. Kohärentes Licht, welches sich aus Wellenlängen einer festgelegten Bandbreite zusammensetzt, wird auf einen semitransparenten Spiegel gesendet.  Dabei wird ein Teil des Lichtes transmittiert und der andere Teil reflektiert. Einer dieser entstehenden Teilstrahlen trifft auf einen Spiegel, an dem er reflektiert wird, und wird als Referenzstrahl benutzt. Der andere Teilstrahl wird auf die zu untersuchende Probe gesendet. Dringt dieser Probenstrahl in die Probe ein, so kann er an den Probenschichten reflektiert werden. \\
Trifft der reflektierte Probenstrahl mit dem Referenzstrahl am semitransparenten Spiegel wieder zusammen, interferieren die beiden Teilstrahlen, wobei die Form des Interferenzmusters von der Probentiefe abhängt, bei der die Reflexion des Probenstrahls stattfindet, und von einem Detektor analysiert wird. Während der Analyse überlagern sich mehrere Signale, die durch Reflexion in unterschiedlichen Probentiefen entstehen. Durch Fourier-Transformation werden diese Signale getrennt, wodurch die komplette Information über das Tiefenprofil der Probe extrahiert werden kann. \\
% Die Optische Kohärenztomographie (OCT) basiert auf dem Prinzip des Michelson-In\-ter\-fe\-ro\-me\-ters~\cite{OCT}. Kohärentes Licht, welches sich aus Wellenlängen einer festgelegten Bandbreite zusammensetzt, wird auf einen semitransparenten Spiegel gesendet. Dabei wird ein Teil des Lichtes transmittiert und der andere Teil reflektiert. Einer dieser entstehenden Teilstrahlen trifft auf einen Spiegel, an dem er reflektiert wird, und wird als Referenzstrahl benutzt. Der andere Teilstrahl wird auf die zu untersuchende Probe gesendet. Dringt dieser Probenstrahl in die Probe ein, so tritt er mit den Atomen und Molekülen des Probenmaterials in Wechselwirkungen und kann dabei gestreut, absorbiert oder reflektiert werden. \\
% Erfährt der Probenstrahl eine Reflexion innerhalb der Probe, so trifft dieser mit dem Referenzstrahl an dem semitransparenten Spiegel wieder zusammen. Die beiden Teilstrahlen werden von dem Spiegel in dieselbe Richtung transmittiert beziehungsweise reflektiert. Dabei interferieren die beiden Teilstrahlen und werden auf einen Detektor geleitet. Dieser analysiert die Intensität des Signals als Funktion der Wellenlänge. Aus dieser Verteilung lässt sich auf die Probentiefe schlie{\ss}en, bei der die Reflexion des Probenstrahls stattfindet. Während der Analyse überlagern sich jedoch mehrere Signale, die durch Reflexionen in verschiedenen Probentiefen entstehen, sodass diese zunächst voneinander getrennt werden müssen. Dazu kann die Fourier-Transformation genutzt werden, welche diese Signale trennt und somit direkt die komplette Information über das Tiefenprofil der Probe liefert. Durch Verschieben der Apparatur entlang der Probe können somit zwei- oder dredimensionale Bilder der Probe erstellt werden. \\
Eines der Hauptanwendungsgebiete der OCT findet sich in der Medizin wieder. Es kann dazu genutzt werden die menschliche Retina auf Krankheiten zu untersuchen. Die Retina besitzt eine komplizierte, mehrschichtige Struktur und Erkrankungen der Retina können in jeder Schicht auftreten. Daher ist die Aufnahme eines Tiefenprofils der Retina von gro{\ss}em medizinischen Interesse. \\
In der vorliegenden Projektarbeit werden OCT Aufnahmen der Retina analysiert, wobei der Datensatz von Ref.~\cite{Dataset} verwendet wird. Die darin enthaltenen Aufnahmen lassen sich in 4 Klassen unterteilen. Drei dieser Klassen beinhalten Aufnahmen von erkrankten Retinas. Eine der Krankheiten ist die choroidale Neovaskularisation (CNV), bei der sich irreguläre Blutgefä{\ss}e in der Retina bilden, aus welchen Flüssigkeiten treten und somit zu Schwellungen der Retina führen können. Drusen (DRUSEN) bezeichnen Abfallprodukten aus der zentralen Retina, die sich unterhalb der Retina ansammeln. Das diabetische Makulaödem (DME) ist eine Schwellung der Makula, die im Zuge der Bildung von kleinen Gefä{\ss}ausbuchtungen in der Retina entsteht. Die vierte Klasse (NORMAL) beinhaltet Retina Aufnahmen, die keine dieser Erkrankungen aufweisen. \\
% Während drei Klassen Aufnahmen beinhalten, welche Erkrankungen der Retina beinhalten, zeigt die vierte Klasse Retina Aufnahmen, die keiner der drei Erkrankungen aufweist und im Folgenden als NORMAL bezeichnet wird. Eine der Krankheiten ist die choroidale Neovaskularisation (CNV), bei der sich irreguläre Blutgefä{\ss}e in der Retina bilden, aus welchen Flüssigkeiten treten und somit zu Schwellungen der Retina führen können. Drusen (DRUSEN) bezeichnen Abfallprodukten aus der zentralen Retina, die sich unterhalb der Retina ansammeln. Das diabetische Makulaödem (DME) ist eine Schwellung der Makula, die im Zuge der Bildung von kleinen Gefä{\ss}ausbuchtungen in der Retina entsteht.  \\
Im Bereich der Bildklassifizierung gibt es bereits zahlreiche Beispiele, bei denen Methoden des maschinellen Lernens eingesetzt werden und gute Ergebnisse erzielen. In dem Projektbericht wird eine Methode basierend auf überwachtem, maschinellem Lernen vorgestellt, die es ermöglicht diese OCT Aufnahmen zu klassifizieren. Diese Methode verwendet dabei ein tiefes faltendes neuronales Netz (CNN) und wird mit einer Methode verglichen, die auf ein neuronales Netz (DNN) zurückgreift, welches aus flachen dichten Lagen besteht. 