%						%
%	packages einbinden	%
%						%

\usepackage[utf8]{inputenc}
%\usepackage{mathptmx}
\usepackage[T1]{fontenc}
\usepackage{amsmath}
\usepackage{amsfonts}
\usepackage{amstext}
\usepackage{amssymb}
\usepackage{amsthm}
\usepackage{mathrsfs}
\usepackage[onehalfspacing]{setspace}

\usepackage{color}
 \definecolor{middlegray}{rgb}{0.5,0.5,0.5}
 \definecolor{lightgray}{rgb}{0.8,0.8,0.8}
 \definecolor{orange}{rgb}{0.8,0.3,0.3}
 \definecolor{yac}{rgb}{0.6,0.6,0.1}
 \definecolor{darkgreen}{rgb}{0,0.5,0}
\definecolor{darkyellow}{rgb}{0.89, 0.82, 0.04}

\usepackage{listings} 			%Code aus zB Python hinzufügen
\lstset{breaklines = true, keywordstyle = 	\bfseries\ttfamily\color{orange}, basicstyle = \scriptsize\ttfamily,
 literate=
  {á}{{\'a}}1 {é}{{\'e}}1 {í}{{\'i}}1 {ó}{{\'o}}1 {ú}{{\'u}}1
  {Á}{{\'A}}1 {É}{{\'E}}1 {Í}{{\'I}}1 {Ó}{{\'O}}1 {Ú}{{\'U}}1
  {à}{{\`a}}1 {è}{{\`e}}1 {ì}{{\`i}}1 {ò}{{\`o}}1 {ù}{{\`u}}1
  {À}{{\`A}}1 {È}{{\'E}}1 {Ì}{{\`I}}1 {Ò}{{\`O}}1 {Ù}{{\`U}}1
  {ä}{{\"a}}1 {ë}{{\"e}}1 {ï}{{\"i}}1 {ö}{{\"o}}1 {ü}{{\"u}}1
  {Ä}{{\"A}}1 {Ë}{{\"E}}1 {Ï}{{\"I}}1 {Ö}{{\"O}}1 {Ü}{{\"U}}1
  {â}{{\^a}}1 {ê}{{\^e}}1 {î}{{\^i}}1 {ô}{{\^o}}1 {û}{{\^u}}1
  {Â}{{\^A}}1 {Ê}{{\^E}}1 {Î}{{\^I}}1 {Ô}{{\^O}}1 {Û}{{\^U}}1
  {œ}{{\oe}}1 {Œ}{{\OE}}1 {æ}{{\ae}}1 {Æ}{{\AE}}1 {ß}{{\ss}}1
  {ű}{{\H{u}}}1 {Ű}{{\H{U}}}1 {ő}{{\H{o}}}1 {Ő}{{\H{O}}}1
  {ç}{{\c c}}1 {Ç}{{\c C}}1 {ø}{{\o}}1 {å}{{\r a}}1 {Å}{{\r A}}1
  {€}{{\euro}}1 {£}{{\pounds}}1 {«}{{\guillemotleft}}1
  {»}{{\guillemotright}}1 {ñ}{{\~n}}1 {Ñ}{{\~N}}1 {¿}{{?`}}1
}
\newcommand*\lstinputpath[1]{\lstset{inputpath=#1}}
\lstinputpath{Auswertung_final}

\usepackage{bibgerm}
\usepackage[ngerman]{babel}
%\usepackage{polyglossia}
%\setdefaultlanguage{english}
\usepackage{caption}
\usepackage{cite}
\usepackage{url}
%\usepackage{cleveref}


\usepackage{textcomp}


\usepackage{longtable}
\usepackage{subfigure}
\usepackage{icomma}
\usepackage{esdiff}
%\usepackage{subcaption}
\usepackage[separate-uncertainty = true,multi-part-units=single,decimalsymbol=comma]{siunitx} % SI-unitx
\usepackage{hyperref}

\usepackage{float}
\restylefloat{figure}

\usepackage{makeidx}
\usepackage{graphicx}
\floatplacement{figure}{htbp}
\floatplacement{table}{htbp}
\usepackage{grffile}
% allow figures to be placed in the running text by default:
\usepackage{scrhack}
\usepackage[left=2.4cm,right=2.4cm,top=3.3cm,bottom=3.3cm]{geometry}
\usepackage{multirow}

\newcommand{\RM}[1]{\MakeUppercase{\romannumeral #1{}}} %römische Zahlen
\renewcommand{\Re}{\operatorname{Re}}

\catcode`\_=\active
\def_#1{\sb{\text{#1}}} %Für aufrechte Indices

\makeatletter
\catcode`\_\active

\def_#1{\sb{\operator@font#1}}
\makeatother
\parindent0cm
\setlength{\footskip}{15pt}
